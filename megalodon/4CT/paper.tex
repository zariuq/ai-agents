\documentclass[11pt]{article}

\usepackage{amsmath,amssymb,amsthm}
\usepackage{hyperref}
\usepackage{algorithm}
\usepackage{algpseudocode}

\newtheorem{theorem}{Theorem}[section]
\newtheorem{lemma}[theorem]{Lemma}
\newtheorem{corollary}[theorem]{Corollary}
\newtheorem{proposition}[theorem]{Proposition}
\newtheorem{definition}[theorem]{Definition}
\newtheorem{remark}[theorem]{Remark}

\title{Kernel-Verified Refutation of a Claimed Four Color Theorem Proof:\\
A Machine-Checked Analysis of Goertzel's Purification Approach}

\author{Verified by Megalodon Proof Kernel}

\date{\today}

\begin{document}

\maketitle

\begin{abstract}
We present machine-verified proofs demonstrating a critical error in a claimed
proof of the Four Color Theorem attributed to Goertzel. The proof attempts to
use face generators and Kempe chain swapping operations to establish that
certain boundary chains span a required vector space. We prove formally that
Lemma 4.3 of the claimed proof is false: the symmetric difference operation
on face generators produces interior-supported chains rather than
boundary-supported chains as claimed. This error propagates through the
proof structure, blocking the entire purification-based approach. All proofs
are kernel-verified in the Megalodon theorem prover (Exit: 0).
\end{abstract}

\section{Introduction}

The Four Color Theorem (4CT) states that any planar map can be colored with
at most four colors such that no two adjacent regions share the same color.
First proven by Appel and Haken in 1976 using computer assistance, and later
formalized in Coq by Gonthier et al., the theorem has a rich history including
Kempe's flawed 1879 proof.

This paper analyzes a claimed proof of 4CT that uses algebraic techniques
involving face generators over the group $\mathbb{F}_2^2$, Kempe chain
operations, and linear algebra arguments. We demonstrate that a key lemma
(Lemma 4.3) in this approach contains a fundamental mathematical error.

\subsection{Main Contribution}

We provide kernel-verified proofs in the Megalodon theorem prover showing:

\begin{enumerate}
\item \textbf{Lemma 4.3 is false}: The claimed equality
$X^f_{\alpha\beta}(C) \oplus X^f_{\alpha\beta}(C^R) = \gamma \cdot 1_R$
(boundary) is wrong. The correct result is
$\gamma \cdot 1_{A \cup A'}$ (interior).

\item \textbf{The error is fundamental}: The symmetric difference of
$(R \cup A)$ and $(R \cup A')$ equals $A \cup A'$, not $R$, because $R$
is common to both sets and therefore cancels.

\item \textbf{Cascade failure}: This error prevents instantiation of the
abstract Lemma 4.4, blocking the entire span argument (Theorems 4.9, 4.10).
\end{enumerate}

\section{Background}

\subsection{Color Algebra}

Colors form the group $G = \mathbb{F}_2^2 = \{0, r, b, p\}$ under XOR, where
$p = r \oplus b$. The key property is self-inverse: $c \oplus c = 0$ for
all colors $c$.

\subsection{Face Generators}

For a face $f$ with boundary $\partial f$, and a two-color pair $(\alpha, \beta)$,
the face generator is defined as:
\[
X^f_{\alpha\beta}(C) := \bigoplus_{R \subset \partial f \cap (\alpha\beta)}
\gamma \cdot 1_{R \cup A_R}
\]
where:
\begin{itemize}
\item $R$ is a maximal $\alpha\beta$-colored run on $\partial f$
\item $A_R$ is one of the two complementary arcs completing the Kempe cycle
\item $\gamma = \alpha \oplus \beta$ is the third color
\end{itemize}

\subsection{Kempe Cycles}

A Kempe cycle $D$ containing run $R$ decomposes as $D = R \cup A \cup A'$ where:
\begin{itemize}
\item $R$ is on the face boundary (the run itself)
\item $A, A'$ are the two complementary arcs (interior to the disk)
\item $R \cap A = R \cap A' = A \cap A' = \emptyset$ (pairwise disjoint)
\end{itemize}

\section{The Mathematical Error}

\subsection{The Claimed Result (Lemma 4.3)}

The claimed proof asserts:
\begin{quote}
``For a single run $R$ with Kempe cycle $D = R \cup A \cup A'$, the per-run
difference is:
\[
X^f_{\alpha\beta}(C) \oplus X^f_{\alpha\beta}(C^R) = \gamma \cdot 1_R
\]
where $C^R$ denotes the coloring after Kempe-switching on $D$.''
\end{quote}

\subsection{The Correct Result}

\begin{theorem}[Per-Run XOR is Interior-Only]
\label{thm:interior}
Given the decomposition $D = R \cup A \cup A'$ with pairwise disjoint sets,
and face generator contributions $\gamma \cdot 1_{R \cup A}$ in $C$ and
$\gamma \cdot 1_{R \cup A'}$ in $C^R$:
\[
(\gamma \cdot 1_{R \cup A}) \oplus (\gamma \cdot 1_{R \cup A'})
= \gamma \cdot 1_{A \cup A'}
\]
\end{theorem}

\begin{proof}
The XOR of indicator chains equals the indicator of the symmetric difference:
\[
(\gamma \cdot 1_S) \oplus (\gamma \cdot 1_T) = \gamma \cdot 1_{S \triangle T}
\]
where $S \triangle T = (S \setminus T) \cup (T \setminus S)$.

For $S = R \cup A$ and $T = R \cup A'$:
\begin{align*}
(R \cup A) \triangle (R \cup A') &= [(R \cup A) \setminus (R \cup A')]
\cup [(R \cup A') \setminus (R \cup A)] \\
&= [A \setminus R \setminus A'] \cup [A' \setminus R \setminus A] \\
&= A \cup A' & \text{(by pairwise disjointness)}
\end{align*}

The key observation: $R$ is contained in both $(R \cup A)$ and $(R \cup A')$,
so it lies in their intersection and therefore cancels in the symmetric
difference.
\end{proof}

\begin{corollary}[Boundary Excluded]
\label{cor:boundary}
For any $x \in R$: $x \notin (R \cup A) \triangle (R \cup A')$.
\end{corollary}

\begin{proof}
If $x \in R$, then $x \in R \cup A$ and $x \in R \cup A'$, so $x$ is in
both sets and therefore not in their symmetric difference.
\end{proof}

\subsection{Formal Verification}

These results are machine-verified in Megalodon with the following theorems:

\begin{verbatim}
Theorem symm_diff_equals_interior :
  forall x:set, x :e symm_diff_result <-> x :e interior.

Theorem boundary_excluded_from_symm_diff :
  forall x:set, x :e R -> x /:e symm_diff_result.

Theorem goertzel_claim_false :
  ~(symm_diff_result = R).
\end{verbatim}

All proofs verify with Exit: 0.

\section{Cascade of Consequences}

\subsection{Lemma 4.4 Cannot Be Instantiated}

The abstract Lemma 4.4 requires constructing ``SwitchData'' such that:
\[
\text{chainXor}(\text{base}, \text{switched}(R)) = \gamma \cdot 1_R
\]

\begin{theorem}[Instantiation Impossible]
There exists no instantiation of Goertzel's face generators satisfying
the SwitchData requirement.
\end{theorem}

\begin{proof}
Suppose such an instantiation exists. Then for any $a \in A$ (interior arc):
\begin{enumerate}
\item $a \in A \cup A' = $ symm\_diff\_result (by Theorem~\ref{thm:interior})
\item The requirement claims symm\_diff\_result $= R$
\item Therefore $a \in R$
\item But $R \cap A = \emptyset$ by hypothesis
\item Contradiction.
\end{enumerate}
\end{proof}

\subsection{Blocked Theorems}

Without the correct instantiation of Lemma 4.4, the following are not proven:
\begin{itemize}
\item Lemma 4.4: $B^f_{\alpha\beta} \in \text{span}(G)$
\item Lemma 4.8: Orthogonality forcing with $\tilde{B}_{\alpha\beta}(S)$
\item Theorem 4.9/4.10: $W_0(H) \subseteq \text{span}(G)$
\end{itemize}

\section{Discussion}

\subsection{Nature of the Error}

The error is not a typo or edge case. It is a fundamental misunderstanding
of how symmetric differences work with sets sharing a common part. The
claim ``XOR gives the boundary $R$'' is the exact opposite of what happens:
XOR cancels the common boundary and preserves the differing interior.

\subsection{Relation to Kempe's Original Error}

Interestingly, this error is distinct from Kempe's original 1879 mistake
(discovered by Heawood in 1890). Kempe's error involved overlapping Kempe
chains interfering during sequential swaps. The error analyzed here is
purely algebraic: the symmetric difference calculation is simply wrong.

\subsection{What This Does Not Claim}

We do \textbf{not} claim:
\begin{itemize}
\item The Four Color Theorem is false (it is proven)
\item No Kempe-chain-based proof can work
\item The abstract lemmas are wrong (they are valid)
\end{itemize}

We \textbf{do} claim: This specific purification mechanism with these
specific face generator definitions cannot work.

\section{Verification Artifacts}

All proofs are available in machine-checkable Megalodon format:

\begin{itemize}
\item \texttt{lemma43\_refutation.mg}: Core symmetric difference theorem
\item \texttt{cascade\_analysis.mg}: Consequence analysis and impossibility
\item \texttt{blocker1\_full.mg}: XOR domain characterization
\item \texttt{blocker2\_full.mg}: Adjacent chain member constraints
\item \texttt{blocker3\_full.mg}: Kempe chain edge constraints
\end{itemize}

Verification command:
\begin{verbatim}
./bin/megalodon -I examples/egal/PfgEMay2021Preamble.mgs <file>.mg
\end{verbatim}

Exit code 0 indicates successful verification.

\section{Conclusion}

We have provided kernel-verified proofs that Lemma 4.3 in a claimed 4CT
proof is mathematically incorrect. The error---confusing boundary support
with interior support in symmetric difference calculations---is fundamental
and cannot be fixed by minor modifications. The entire purification-based
proof avenue is blocked by this error.

\section*{Acknowledgments}

The formal verification was performed using the Megalodon theorem prover
with the PfgEMay2021 preamble.

\begin{thebibliography}{9}

\bibitem{appel1976}
K. Appel and W. Haken.
\newblock Every planar map is four colorable.
\newblock {\em Bull. Amer. Math. Soc.}, 82:711--712, 1976.

\bibitem{gonthier2008}
G. Gonthier.
\newblock Formal proof---the four-color theorem.
\newblock {\em Notices Amer. Math. Soc.}, 55(11):1382--1393, 2008.

\bibitem{heawood1890}
P.J. Heawood.
\newblock Map colour theorem.
\newblock {\em Quarterly Journal of Mathematics}, 24:332--339, 1890.

\bibitem{kempe1879}
A.B. Kempe.
\newblock On the geographical problem of the four colours.
\newblock {\em Amer. J. Math.}, 2:193--200, 1879.

\end{thebibliography}

\end{document}
